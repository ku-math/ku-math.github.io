\documentclass[12pt]{article}
\usepackage{amsmath, amssymb} 
\usepackage{geometry}
\geometry{a4paper, margin=1in}


\begin{document}

\section{Porblems set 1(Real Numbers)}

\begin{enumerate}

    \item Prove the following statements are true in every ordered field.

    \begin{enumerate}
        \item If $x > 0$ then $-x < 0$, and vice versa.
        \item If $x > 0$ and $y < z$ then $xy < xz$.
        \item If $x < 0$ and $y < z$ then $xy > xz$.
        \item If $x \neq 0$ then $x^2 > 0$. In particular, $1 > 0$.
        \item If $0 < x < y$ then $0 < 1/y < 1/x$.
    \end{enumerate}

    \item Let $S\subset\mathbb{R}$ and $a\in\mathbb{R}$. Prove that
    \begin{enumerate}
        \item $\sup (a+S)=a+\sup S$
        \item If $a>0$ then $\sup (aS)=a\sup S$
        \item If $a>0$ then $\sup (aS)=a\inf S$
    \end{enumerate}
    \item Prove that supremum property implies infimum property.
    \item Use Archimedean Property to prove the following
    \begin{enumerate}
        \item If $S := \{1/n : n \in \mathbb{N}\}$, then $\inf S = 0$.
        \item If $t > 0$, there exists $n_t \in \mathbb{N}$ such that $0 < 1/n_t < t$.
        \item If $y > 0$, there exists $n_y \in \mathbb{N}$ such that $n_y - 1 \leq y \leq n_y$.
        \item If \( S := \left\{ \frac{1}{n} - \frac{1}{m} : n, m \in \mathbb{N} \right\}, \) find \( \inf S \) and \( \sup S \).
    \end{enumerate}
    \item Prove that between any two real numbers there is an irrational number.
    \item Let \( I_n := [0, 1/n] \) for \( n \in \mathbb{N} \). Prove that
    \[
    \bigcap_{n=1}^{\infty} I_n = \{0\}.
    \]

    \item Let \( J_n := (0, 1/n) \) for \( n \in \mathbb{N} \). Prove that
    \[
    \bigcap_{n=1}^{\infty} J_n = \emptyset.
    \]

    \item Let \( K_n := (n, \infty) \) for \( n \in \mathbb{N} \). Prove that
    \[
    \bigcap_{n=1}^{\infty} K_n = \emptyset.
    \]

\end{enumerate}

\newpage

\section{Problem set 2(Sequences)}

\begin{enumerate}
     \item Find $K(\epsilon ) $ for the following sequences 
    \begin{enumerate}
        \item $\frac{1}{5n+1} ~;~ \epsilon = \frac{1}{6} $
        \item $\frac{2}{n^3}+5 ~; ~\epsilon = \frac{1}{8}$
        \item $\frac{3n}{2n+1} ~; ~\epsilon = \frac{1}{10}$
        \item$\frac{n^2 + n }{2n^2 -1} ~; ~\epsilon = \frac{1}{5}$
        \item$(-1)^n \frac{1 }{n} ~; ~\epsilon = \frac{1}{61}$
    \end{enumerate}

    \item Porve that limit of a sequence if it exist is unique.
    \item Evaluate the following limits (also prove it using the definition of limit):
    \begin{enumerate}
        \item \( \lim_{n \to \infty} \frac{3n}{2n+1} \)
        \item \( \lim_{n \to \infty} \frac{n}{n^2 + 1} \)
        \item \( \lim_{n \to \infty} (\sqrt{n+1} - \sqrt n) \)
    \end{enumerate}
    \item Let $x_n,y_n$ be two sequences such that $x_n \to x$ and $y_n \to y$. Prove that:
    \begin{enumerate}
        \item \( x_n + y_n \to x + y \)
        \item \( x_n - y_n \to x - y \)
        \item \( x_n y_n \to xy \)
        \item \( x_n / y_n \to x/y \) if \( y \neq 0 \)
    \end{enumerate}
    \item Prove that if $x_n\to x$ then $|x_n|\to|x|$ and $\sqrt{|x_n|}\to\sqrt{|x|}$.
    \item Prove squeeze theroerm:
    If \( x_n \leq y_n \leq z_n \) for all \( n \) and \( x_n \to L \), \( z_n \to L \), then \( y_n \to L \).
    \item Prove that for any real number $x$, there exists a sequence of rational numbers $(x_n)$ such that $x_n \to x$.
\end{enumerate}

\newpage

\section{Problem set 3(Sequences)}
    \begin{enumerate}
        \item Give an example of an unbounded sequence that has a convergent subsequence.
        \item Suppose that $x_n\geq0$ and $(-1)^nx_n$ convereges. Does it imply that $x_n$ converges.
        \item Let $(x_n)$ be a bounded sequence and let $s := \sup\{x_n : n \in \mathbb{N}\}$. Show that if $s \notin \{x_n : n \in \mathbb{N}\}$, then there is a subsequence of $(x_n)$ that converges to $s$.
        \item Show directly that bounded monotone sequence is a cauchy sequence.
        \item Let $Y = (y_n)$ be defined inductively by $y_1 := 1$, $y_{n+1} := \frac{1}{4}(2y_n+3)$ for $n \ge 1$. Show that $\lim Y = 3/2$.
    \end{enumerate}

\end{document}