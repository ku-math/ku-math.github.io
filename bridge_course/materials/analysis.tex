\documentclass[12pt]{article}
\usepackage[utf8]{inputenc}
\usepackage{amsmath, amssymb, amsthm} % For math symbols, environments
\usepackage{geometry} % For page margins
\geometry{a4paper, margin=1in}

% For definitions
\newtheorem{definition}{Definition}[section] % Definitions numbered by section
\newtheorem{theorem}{Theorem}[section]     % Theorems numbered by section


\begin{document}

\begin{center}
    \textbf{\Large Lecture 1: Real Numbers}
\end{center}

\section*{Introduction}
Today, we're going to solidify our understanding of the number line. We'll start from what we know (rational numbers) and introduce the properties that complete the number line, giving us the real numbers. The goal is to understand \textit{why} we need real numbers beyond rationals.

\section{Building Blocks: From Natural Numbers to Rational Numbers}

\begin{itemize}
    \item \textbf{Natural Numbers ($\mathbb{N}$):} Counting numbers $\{1, 2, 3, \dots\}$. Basic arithmetic (addition, multiplication).
    \item \textbf{Integers ($\mathbb{Z}$):} Include negatives and zero $\{\dots, -2, -1, 0, 1, 2, \dots\}$. Allows for subtraction.
    \item \textbf{Rational Numbers ($\mathbb{Q}$):} Fractions $p/q$, where $p \in \mathbb{Z}, q \in \mathbb{N}$. Allows for division (except by zero).
\end{itemize}

\textbf{Key Idea:} $\mathbb{Q}$ forms a \textbf{field} (can add, subtract, multiply, divide) and an \textbf{ordered set} (can compare numbers, $a < b$, $a=b$, or $a > b$).

\section{Ordered Fields: A Foundation for Real Numbers}

\begin{definition}
An \textbf{ordered field} is a set of numbers that behaves like $\mathbb{Q}$ in terms of arithmetic and order.
\begin{itemize}
    \item It's a \textbf{field}: You can add, subtract, multiply, and divide (except by zero), and these operations follow standard rules (associativity, commutativity, etc.).
    \item It's \textbf{ordered}: You can compare any two numbers (one is greater than, less than, or equal to the other), and this order is compatible with arithmetic.
\end{itemize}
\end{definition}

\textbf{Motivation:} We want to build the real numbers on a solid algebraic and order-based foundation. $\mathbb{Q}$ is our prime example of an ordered field.

\section{Gaps in Rationals}

\textbf{Motivation:} While rationals seem to fill the number line, they have "holes."
\begin{itemize}
    \item Consider $\sqrt{2}$. We know $x^2 = 2$ has no rational solution.
    \item Geometrically, $\sqrt{2}$ is the length of the diagonal of a unit square. It exists! But it's not a rational number.
    \item This means there are points on our intuitive number line that aren't represented by rational numbers. We need to "fill these gaps."
\end{itemize}

\section{Completeness: Filling the Gaps}

To fill the gaps, we introduce the idea of "completeness." This relies on the concepts of bounds, supremum, and infimum.

\begin{itemize}
    \item \textbf{Upper Bound:} A number $M$ is an upper bound for a set $S$ if all elements in $S$ are less than or equal to $M$.
    \item \textbf{Supremum (Least Upper Bound, LUB):} The smallest of all upper bounds. If a set has a supremum, it's like finding the "edge" of the set from above.
    \begin{itemize}
        \item \textbf{Motivation:} If a set of numbers is growing but doesn't go on forever (it's "bounded above"), it should "converge" to a specific value, its "edge." In $\mathbb{Q}$, this doesn't always happen (e.g., $\{x \in \mathbb{Q} \mid x^2 < 2\}$ should have $\sqrt{2}$ as its edge, but $\sqrt{2}$ isn't rational).
    \end{itemize}
    \item \textbf{Lower Bound:} A number $m$ is a lower bound for a set $S$ if all elements in $S$ are greater than or equal to $m$.
    \item \textbf{Infimum (Greatest Lower Bound, GLB):} The largest of all lower bounds.
\end{itemize}

\begin{definition}[\textbf{The Supremum Property (Completeness Axiom)}]
Every non-empty set of real numbers that is bounded above has a supremum \textit{that is also a real number}.
\end{definition}

\textbf{Motivation:} This is the crucial axiom that fills the "holes." It guarantees that if a set of numbers is "approaching" a value from below (and is bounded), that "limiting" value \textit{must} exist within our number system. This is what allows for concepts like limits, continuity, and convergence in calculus.

\section{Defining the Real Numbers ($\mathbb{R}$)}

\begin{theorem}
    There exist a unique ordered field that contains the rational numbers $\mathbb{Q}$ and satisfies the \textbf{supremum property} which we denote as $\mathbb{R}$.
\end{theorem}

We are saying: take the rational numbers, add the "completeness" property to them (which means no more holes!), and what you get is the real numbers. The "uniqueness" means that this set of properties perfectly defines $\mathbb{R}$ -- there's only one mathematical structure that fits.

\section{Fundamental Properties of $\mathbb{R}$}

These properties are direct consequences of the completeness of $\mathbb{R}$.

\subsection{Archimedean Property}

\begin{theorem}
For any positive real number $x$, there exists a natural number $n \in \mathbb{N}$ such that $n > x$.
\end{theorem}
This property formalizes the idea that the natural numbers "stretch" indefinitely. No matter how large a real number $x$ is, you can always count past it. It also implies that for any tiny positive number $\epsilon$, you can find a fraction $1/n$ that's even smaller. This is crucial for convergence proofs (e.g., showing a sequence approaches a limit).

\subsection{Density of Rationals}

\begin{theorem}
If $x, y \in \mathbb{R}$ and $x < y$, then there exists a rational number $q \in \mathbb{Q}$ such that $x < q < y$.
\end{theorem}
This tells us that even though $\mathbb{Q}$ has "holes," it's incredibly "dense" within $\mathbb{R}$. No matter how close two real numbers are, you can always "squeeze" a rational number between them. This means rationals can approximate any real number as closely as we desire, which is fundamental for many areas of analysis.

\subsection{Nested Interval Property}
\begin{theorem}
Let $\{I_n\}_{n=1}^{\infty}$ be a sequence of closed and bounded intervals $I_n = [a_n, b_n]$ such that $I_{n+1} \subseteq I_n$ for all $n \in \mathbb{N}$ (the intervals are nested). Then there exists a real number $x$ such that $x \in I_n$ for every $n \in \mathbb{N}$.
\end{theorem}
This property is a direct consequence of the completeness of the real numbers and serves as a powerful tool in analysis. It essentially states that if you keep "zooming in" on the number line by creating smaller and smaller nested intervals, you will always converge to a single, unique real number. This is crucial for proving the existence of solutions, roots, or limits in various mathematical contexts, ensuring that there are no "gaps" left unaddressed by this process of successive approximation. It underpins many fundamental theorems in real analysis, such as the Bolzano-Weierstrass Theorem.

\subsection{Uncountability of Real Numbers ($\mathbb{R}$)}
\begin{theorem}
The set of real numbers $\mathbb{R}$ is uncountable.
\end{theorem}

This profound theorem reveals that not all infinities are the same size. Despite the density of rationals, the set of real numbers is fundamentally "larger," meaning the "gaps" in the rational number line, filled by irrational numbers, constitute an overwhelmingly vast collection of points. The uncountability of $\mathbb{R}$ highlights its essential completeness, a property rigorously utilized through the Nested Interval Property.

\newpage

\section{Problems for Practice}

\begin{enumerate}

    \item Prove the following statements are true in every ordered field.

    \begin{enumerate}
        \item If $x > 0$ then $-x < 0$, and vice versa.
        \item If $x > 0$ and $y < z$ then $xy < xz$.
        \item If $x < 0$ and $y < z$ then $xy > xz$.
        \item If $x \neq 0$ then $x^2 > 0$. In particular, $1 > 0$.
        \item If $0 < x < y$ then $0 < 1/y < 1/x$.
    \end{enumerate}

    \item Let $S\subset\mathbb{R}$ and $a\in\mathbb{R}$. Prove that
    \begin{enumerate}
        \item $\sup (a+S)=a+\sup S$
        \item If $a>0$ then $\sup (aS)=a\sup S$
        \item If $a>0$ then $\sup (aS)=a\inf S$
    \end{enumerate}
    \item Prove that supremum property implies infimum property.
    \item Use Archimedean Property to prove the following
    \begin{enumerate}
        \item If $S := \{1/n : n \in \mathbb{N}\}$, then $\inf S = 0$.
        \item If $t > 0$, there exists $n_t \in \mathbb{N}$ such that $0 < 1/n_t < t$.
        \item If $y > 0$, there exists $n_y \in \mathbb{N}$ such that $n_y - 1 \leq y \leq n_y$.
        \item If \( S := \left\{ \frac{1}{n} - \frac{1}{m} : n, m \in \mathbb{N} \right\}, \) find \( \inf S \) and \( \sup S \).
    \end{enumerate}
    \item Prove that between any two real numbers there is an irrational number.
    \item Let \( I_n := [0, 1/n] \) for \( n \in \mathbb{N} \). Prove that
    \[
    \bigcap_{n=1}^{\infty} I_n = \{0\}.
    \]

    \item Let \( J_n := (0, 1/n) \) for \( n \in \mathbb{N} \). Prove that
    \[
    \bigcap_{n=1}^{\infty} J_n = \emptyset.
    \]

    \item Let \( K_n := (n, \infty) \) for \( n \in \mathbb{N} \). Prove that
    \[
    \bigcap_{n=1}^{\infty} K_n = \emptyset.
    \]

\end{enumerate}


\end{document}